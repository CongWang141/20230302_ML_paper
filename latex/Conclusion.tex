In this research paper, we re-examine the longstanding yet ever-relevant question in empirical asset pricing of predicting stock returns. Our study adds to the existing body of empirical asset pricing literature in several significant ways. Firstly, predicting stock returns is a challenging task, as it involves accounting for the interaction effects between various predictors and the non-linear relationships present in the underlying mapping function. Traditional linear models are unable to account for these complexities, while modern neural network models have shown to possess superior capabilities in capturing these interactions and non-linearities. This study offers a comprehensive guide on utilizing deep neural network models to predict stock returns. The guide covers crucial aspects, such as determining the appropriate model settings to achieve specific objectives, understanding the impact of each hyperparameter on the model's performance, and optimizing hyperparameters to enhance the predictive accuracy of the model.

Secondly, our study distinguishes itself from most empirical studies in this area by examining the predictability of different versions of stock abnormal returns and stock excess return. Specifically, we define stock abnormal returns as the alpha of CAPM, Fama-Frech 3 factors, and Fama-Frech 5 factors model in predicting excess returns, representing the difference between the model's expected stock returns and the actual stock excess returns. We find that, using the same information set, the model shows nearly the same predicting power in predicting different measures of stock returns. In fact, our results indicate that the out-of-sample R-squared values for predicting stock abnormal and excess are almost identical.

Thirdly, we incorporate macroeconomic variables CFNAI and investor sentiment data to capture the macroeconomic conditions. Our analysis reveals a strong interaction effect between these macroeconomic index variables and firm characteristic variables when predicting stock excess return. Adding these macroeconomic variables, along with the 49 firm characteristic variables to the neural network model, significantly improves the out-of-sample R-Squared value for predicting stock excess returns. However, we observe that the interaction effects are not as prominent when predicting stock abnormal return. This is primarily because the factors models has already eliminated the time trend associated with macroeconomic flatulation in the stock abnormal return.

Finally, we divide the entire dataset into three subperiods, namely recession, normal, and expansion periods based on the CFNAI index. The significant interaction effects observed between the macroeconomic index variables and firm characteristic variables indicate that stock returns vary across different macroeconomic conditions. Our analysis reveals that the model-predicted portfolios generate higher mean returns and Sharpe ratios during recession period when the market is more volatile, while they perform relatively lower during normal and expansion periods when the market is less volatile. This finding provides valuable insights into identifying the optimal market timing for investment decisions.