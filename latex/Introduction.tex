Machine learning (ML) methods have become increasingly popular in empirical asset pricing research due to their ability to analyze high-dimensional financial data and their high accuracy in predicting future asset returns. \citet*{gu2020empirical} provide a comprehensive overview of ML methods for predicting stock excess returns and find that neural network models outperform most other methods. This paper builds upon their work by providing a detailed guide on how to use deep neural network models to predict stock returns. It includes a general description of how to select objective and activation functions, apply learning rate schedule to speed up learning and achieve high prediction accuracy, use regularization to avoid overfitting, improve predictions with ensemble learning, and compute feature importance using novel SHAP values.

This empirical study focuses exclusively on the US stock market and covers the period from 1971 to 2021. The study employs 49 stock characteristic variables to capture stock fundamentals and also  includes macroeconomic variables such as CFNAI index and investor sentiment to summarize the economy's overall conditions as predictors. Unlike previous studies that predict only stock excess returns, this paper distinguishing itself from others by comparing the model's performance in predicting multi-measures of stock returns. Stock excess return is defined as the difference between a stock's actual return and the one-month treasury bill rate, while stock abnormal return refers to the alpha returns obtained from factor models. Abnormal returns are useful for evaluating individual stocks and portfolios and indicate whether a stock has outperformed or underperformed the market expectation. Given the research spanned over five decades, the factor models have evolved from CAPM to Fama-French 5 factors. Since it is unclear which factor models investors use during each time period, this paper extensively compares abnormal returns derived from multiple factor models.

Prior to the main analysis with the neural network model, we conduct an univariate long-short portfolio analysis. Stocks are sorted into deciles based on the 45 firm-specific characteristic feature variables, and long-short portfolios are constructed by long the top decile and short the bottom decile. We find that most of our stock characteristic feature variables have predicting power, with the portfolios achieving high mean returns and Sharpe ratios than average. Next, the full sample is divided into three subsamples based on the CFNAI index, named as recession, normal, and expansion periods. We confirm that there are interaction effects between stock characteristic variables and macroeconomic conditions in predicting stock returns. Portfolios realized higher mean return and Sharpe ratio in the recession periods, and lower mean return and Sharpe ratio in the normal period when the market was less volatile regardless of different measures of stock returns. It is important to note that the analysis of univariate long-short portfolios represents only the first step in exploring the relationship between our candidate predictors and stock returns. This approach does not account for the non-linear relationship between predictors and stock returns, nor the interaction effects among each firm-specific characteristic variable.

In the main analysis, we first use only the 49 firm-specific characteristic feature variables to predict stock returns. Our empirical study reveals that the neural network model's performance in predicting abnormal returns is almost equal to its performance in predicting excess returns. The out-of-sample R-Squared value in the testing dataset for predicting abnormal returns, based on the CAPM model, is the highest at 0.923\%, while the lowest R-squared value comes from predicting abnormal returns based on the Fama-French 3 factors model, which is 0.762\%. However, there is not a significant difference in predicting accuracy in terms of R-squared values, as the R-squared values for predicting various measures of stock returns are around 0.8\%.

Next, we incorporate the macroeconomic variables including CFNAI and investor sentiment data, which provide a summary of the overall macroeconomic conditions, into the neural network model. Our analysis demonstrates a notable improvement in out-of-sample performance for predicting stock excess returns. Specifically, the out-of-sample R-Squared value for predicting stock excess returns in the testing sample increased by 79.8\% from 0.811\% to 1.458\% after adding CFNAI data. When both CFNAI and investor sentiment data are added to the model, the out-of-sample R-squared value increased nearly 5 times to 4.005\%. However, the increase of R-squared values is not so profound in the prediction of stock abnormal returns after adding these macroeconomic variables, no matter which factor models is used to derive the abnormal returns. These results suggest that incorporating macroeconomic variables can significantly improve the accuracy of stock excess return predictions, but may not have as much impact on predicting abnormal returns.

The performance of value-weighted portfolios based on predictions varies significantly, with the lowest decile indicating poor performance and the highest decile indicating the best performance, even when predicting only with firm characteristics. The distribution of the portfolios' cumulative abnormal returns is nearly symmetrical, with the spectrum evenly allocated on both sides of the zero line. Investing in the top decile could yield up to 10 times abnormal returns for the entire time period, while investing in the bottom decile could result in a loss of the same magnitude. In contrast to the spectrum of abnormal stock returns, the portfolios' cumulative excess returns exhibit an upward trend for most deciles. Investing in the top decile could yield nearly 12 times excess return, while investing in the bottom decile could result in a loss of close to 5 times excess return. After incorporating macroeconomic variables, there are slight improvements in distinguishing the performance of different portfolios. This is consistent with the improvements in the R-squared values observed after including these additional variables.

Then, we explore the feature importance for predicting stock returns using the neural network model with all the candidate variables. We use SHAP (SHapley Additive exPlanation) value to measure the relative feature importance. For predicting stock abnormal return, we find that variables with the most predictive power belong to momentum, such as long-term reversal (LRreversal), short-term reversal (STreversal), and mid-term reversal (MRreversal), etc. Variables belonging to trading fraction rank second, such as past trading volume (DolVol) and CAPM beta (Beta), etc. Variables that capture macroeconomic conditions rank fourth after variables from the value and risk groups. However, when it comes to predicting stock excess return, macroeconomic variables become the most important predictors, and the rank of the following groups remains unchanged. There change of feature importance in predicting different measures of stock returns indicating the fundamental difference between stock excess and abnormal returns.

To investigate whether the neural network model's predicted portfolios could achieve a higher mean return and Sharpe ratio under different macroeconomic conditions, we divide the full sample into three subsamples based on the CFNAI index, as we did in the univariate long-short portfolios analysis. These subsamples are defined as recession, normal, and expansion periods. Within each period, we construct long-only portfolios based on the neural network model's predicted return to explore if we could obtain different performance in different time periods. Our analysis revealed that for both stock abnormal and excess returns, the top decile in recession periods yielded the highest mean return and Sharpe ratio when the market is more volatile, while the lowest performance was observed during normal periods when the market is more peaceful, which coincides with the findings in the univariate portfolio analysis. Other portfolios, besides the top deciles, exhibited mixed performance across the three time periods. These findings suggest that the neural network model's predicted portfolios may perform differently under various macroeconomic conditions, and the best market timing for investment may depend on these conditions.