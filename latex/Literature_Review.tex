Machine learning has been increasingly used in the field of empirical asset pricing in recent years. The goal is to develop and implement models that can predict future returns or pricing of assets, exhaustive reviews in this area refer to \citet*{giglio2022factor}, and \citet*{bagnara2022asset}. \citet*{freyberger2020dissecting} use adoptive group LASSO to predict stock excess return nonparametrically and explore which firm characteristics give independent information for the cross-section. As one drawback of linear models such as LASSO and Elastic-net, the interaction effects among feature variables are not accounted. \citet*{bryzgalova2020forest} use decision trees to endogenously group stocks together and select optimal portfolio splits to span the Stochastic Discount Factor. Non-linear tree models not only can accomodate the interaction effects among feature variables but also can predict more interpretable results. 

As \citet*{cochrane2011presidential} bring up in his AFA Presidential Adress, the currently challenge in asset pricing research is the "the curse of high-dimensionality", hence another research direction is to reduce the dimensionality of predictors. \citet*{kelly2019characteristics} propose Instrumented Principal Component Analysis (IPCA) to predict the crose section of returns. They use observable characteristics that instrument for unobservable dynamic factor loadings, in such a way the fomulated latent factors have lower dimension and process time varying character. \citet*{kozak2020shrinking} impose a prior on SDF coefficients that shrinks low-variance principal components of the firm characteristics factors, and a SDF built by smaller number of reduced components performs even better than the four- or five-factor models in the recent literature.

\citet*{gu2020empirical} conduct an empirical study to compare various prediction models. They find that machine learning models, particularly neural network models, can outperform traditional statistical models in terms of accuracy and precision, especially when dealing with high-dimensional and complex financial data. Further research have also beening made by using deep neural network models. \citet*{gu2019autoencoder} propose a new latent conditional asset pricing model following the seminar research by \citet*{kelly2019characteristics}. They retrofit the workhorse unsupervised dimension reduction device by using autoencoder neural networks. \citet*{chen2019deep} use deep neural networks, including feedforward networks, long-short term memory, and adversarial networks, to estimate the pricing kernel model that predicts cross-sectional stock returns based on a large amount of information. The deep neural networks learning method outperforms benchmark approaches in terms of out-of-sample Sharpe ratio.

While most machine learning papers in empirical asset pricing focus on predicting stock excess return, \citet*{kaniel2022machine} examine the predictive power of neural network models for predicting fund abnormal return. Their findings suggest novel and substantial interaction effects between market sentiment and both fund flow and fund momentum. It raises the question of whether stock abnormal return is more predictable than excess return, and whether there are significant interaction effects between macroeconomic conditions and stock returns.

Stock return prediction models have evolved significantly since the last century, predating the incorporation of machine learning methods. The Capital Asset Pricing Model (CAPM), introduced by \citet*{sharpe1964capital}, describes the concept of beta as a measure of an asset's risk relative to the market. Later, \citet*{fama1993common} expanded the model to include two additional factors, size and value, which explain the cross-section of expected returns. The size factor captures the tendency for smaller firms to have higher expected returns, while the value factor captures the tendency for value stocks (i.e., stocks with low price-to-book ratios) to have higher expected returns. More recently, \citet*{fama2015five} added two additional factors to create the 5 factor model. The profitability factor captures the risk associated with a firm's profitability, the investment factor captures the risk associated with a firm's investment policy. In the recent years, researchers add the momentum factor captures the risk associated with a stock's recent price momentum. These factors help capture additional sources of risk and return not covered by the original 3 factors.

Exploring whether a stock's abnormal return (i.e., the return that deviates from the expected return based on factors model) is more predictable than its excess return (i.e., the return that deviates from the risk-free rate) is an intriguing question. Additionally, investigating how macroeconomic variables factor into predicting stock returns can provide valuable insights for both theoretical and empirical research. These questions are worth dissecting and exploring further to provide empirical evidence.